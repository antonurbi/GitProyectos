\documentclass{article}
\usepackage[utf8]{inputenc}
\usepackage[spanish]{babel}

\title{Así han engañado a los chinos}
\author{José Luis Cava}
\date{}

\begin{document}

\maketitle

\section*{Introducción}

Hoy les planteo la siguiente pregunta: ¿Por qué decenas de millones de chinos están descontentos con su gobierno? Para responder a esta pregunta, primero debemos entender el concepto de \textit{hipoteca china}, que comenzó a partir de 1998.

\section*{El concepto de la hipoteca china}

Cuando un ciudadano chino solicita un préstamo hipotecario, no solo responde con el bien hipotecado, sino con todo su patrimonio. A diferencia de lo que sucede en Occidente, donde se paga un pequeño porcentaje del precio de venta final (20-30\%), en China, cuando un ciudadano compra una casa sobre plano, debe pagar el 100\% del precio.

Esto implica que, al firmar el contrato de compra, se paga el 100\%, y se solicita el préstamo hipotecario por el importe total. Desde el momento en que se firma el contrato, aunque no se entregue la vivienda, ya se comienzan a pagar las cuotas de amortización del préstamo hipotecario (capital más intereses).

\section*{Las inmobiliarias en China}

Las inmobiliarias en China están encantadas con este sistema porque reciben el dinero desde el comienzo sin pagar intereses. Con este dinero, suelen comprar otro terreno para iniciar una nueva promoción y así repetir el proceso. Esto crea una especie de \textit{timo piramidal}. Mientras sigan vendiendo inmuebles no construidos, continúan comprando terrenos.

Si en algún momento se ralentiza el proceso o les falta dinero, bajan la calidad de los materiales, y los bancos lo consienten. En Occidente, los bancos suelen seleccionar y vigilar la calidad y moralidad de las inmobiliarias, pero en China no sucede lo mismo.

\section*{El papel del gobierno}

El gobierno tampoco interviene porque se beneficia de este sistema. La inmobiliaria obtiene financiación gratuita, el banco concede una gran cantidad de préstamos hipotecarios, y el gobierno recauda impuestos mientras desarrolla el mercado inmobiliario, alcanzando los objetivos de crecimiento económico del 5\%.

\section*{Cambios recientes}

En los últimos años, debido a la cantidad de estafas, las autoridades chinas han hecho algunos cambios. Ahora no se paga el 100\% del precio de compra de la vivienda sobre plano, sino cuando se llega al estado de \textit{cubre aguas}, lo que debería reducir el número de promociones fallidas.

Además, en lugar de pagar directamente a la inmobiliaria, el dinero se deposita en una cuenta, y una autoridad gubernamental va pagando en función de las certificaciones de obra. Sin embargo, en la práctica, estos cambios no se cumplen, lo que ha generado descontento entre los ciudadanos chinos.

\section*{Medidas monetarias recientes}

El gobierno, en su nuevo paquete de medidas monetarias, ha reducido el tipo de interés de los préstamos hipotecarios que pagan los ciudadanos chinos. A pesar de esto, la situación sigue siendo insostenible, lo que explica el descontento social y la preocupación por relanzar el crecimiento económico en China. Es probable que se necesite un gran paquete de estímulo fiscal.

\section*{Conclusión}

Esta situación ha generado un enorme descontento social en China, que está abocada a aplicar más medidas de estímulo para reactivar su economía.

\end{document}
