\documentclass[12pt]{article}
\usepackage{graphicx}
\usepackage{amsmath}

\title{Análisis del Impacto Económico de un Conflicto Geopolítico: \\ Estrategia Innovadora en el Mercado Energético}
\author{Nombre del Trader}

\begin{document}

\maketitle

\begin{abstract}
Este artículo analiza el impacto económico potencial de un conflicto militar entre Israel e Irán, específicamente centrado en la destrucción de refinerías iraníes. A partir de este escenario, se desarrolla una estrategia de trading en productos refinados, que ofrece una alternativa poco común en el mercado de commodities. El análisis identifica oportunidades en el mercado energético global, destacando la importancia de productos derivados del petróleo como la gasolina y el diésel.
\end{abstract}

\section{Introducción}
El conflicto entre Israel e Irán genera una creciente preocupación por la estabilidad del mercado energético global. Según el análisis económico, la destrucción de las refinerías iraníes provocaría un aumento significativo en los precios de productos refinados, mientras que el impacto en el crudo sería más volátil y menos predecible. Este trabajo examina estos posibles escenarios y propone una estrategia de trading en productos refinados.

\section{Impacto Económico del Conflicto}
Irán produce 3.6 millones de barriles de crudo por día, de los cuales exporta 1.7 millones, principalmente a China. Si Israel ataca las refinerías iraníes, la producción de 2.9 millones de barriles diarios de productos refinados se detendría, provocando un aumento global de los precios de la gasolina y el diésel. Este evento tendría consecuencias tanto para los mercados internos iraníes como para los globales.

\section{Estrategia Financiera}
La estrategia propuesta se basa en la identificación de dislocaciones en el mercado de productos refinados en lugar de centrarse en el mercado de crudo. A continuación, se detalla cómo aprovechar estas oportunidades:

\subsection{Posiciones largas en productos refinados}
Dado que los productos refinados sufrirán un aumento de demanda y precios, se propone tomar posiciones largas en futuros de gasolina y diésel.

\subsection{Trading asimétrico}
Se recomienda un enfoque asimétrico: mantener posiciones largas en productos refinados y posiciones cortas en crudo, aprovechando las diferencias en las dinámicas de oferta y demanda.

\subsection{Spread Trading}
El spread trading entre futuros de crudo y refinados permite obtener beneficios en ambos lados de la ecuación. En momentos de dislocación, este tipo de estrategia puede ser extremadamente rentable.

\section{Conclusión}
La situación geopolítica actual ofrece oportunidades únicas en el mercado de productos refinados. Al centrarse en estrategias menos comunes como el spread trading y el enfoque en productos refinados, se puede lograr una ventaja competitiva significativa. El conflicto entre Israel e Irán será un catalizador clave para estas dislocaciones de mercado, haciendo de esta estrategia una opción viable y rentable.

\end{document}
