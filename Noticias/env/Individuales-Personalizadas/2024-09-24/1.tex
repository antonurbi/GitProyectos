\documentclass{article}
\usepackage{amsmath}
\usepackage{geometry}
\geometry{letterpaper, margin=1in}

\title{Resumen para Inversores: Opciones, Futuros y Otros Derivados}
\author{Inversión Estratégica}

\begin{document}

\maketitle

\section*{Introducción}

Este documento ofrece un resumen sobre derivados, específicamente opciones y futuros, basado en el texto de referencia sobre el tema. El mercado de derivados ofrece oportunidades para diversificar riesgos, pero también involucra complejidades y regulaciones a tener en cuenta.

\section*{Definición de Derivados}

Un derivado es un instrumento financiero cuyo valor \textbf{se deriva} del valor de otra variable subyacente más básica. El derivado no tiene un valor propio, sino que depende de otro activo o indicador. Los subyacentes pueden ser:

\begin{itemize}
    \item \textbf{Activos financieros} como acciones o índices bursátiles.
    \item \textbf{Activos reales} como propiedades.
    \item \textbf{Eventos} como la cantidad de nieve en una temporada de esquí.
\end{itemize}

\section*{Mercados de Derivados}

Los derivados pueden negociarse en dos tipos de mercados:

\begin{itemize}
    \item \textbf{Mercados organizados}: Son altamente regulados, donde los contratos son estandarizados y las transacciones pasan por una cámara de compensación, eliminando el riesgo de contraparte. Nunca ha habido un incumplimiento en derivados negociados en bolsa.
    \item \textbf{Mercado Over-the-Counter (OTC)}: Permite mayor personalización, pero conlleva más riesgo de contraparte. Aunque está menos regulado, las regulaciones están aumentando desde la crisis financiera de 2008-2009.
\end{itemize}

\section*{Forwards y Futuros}

\textbf{Contratos Forward}: Son acuerdos para comprar o vender un activo subyacente en una fecha futura a un precio pactado hoy. Se negocian en el mercado OTC.

\textbf{Contratos de Futuros}: Son similares a los forwards, pero se negocian en mercados organizados. Las diferencias clave entre ambos incluyen:

\begin{itemize}
    \item Los futuros están estandarizados y se negocian en bolsa.
    \item Los forwards son personalizados y se negocian OTC.
    \item Ambos requieren un comprador y un vendedor, y las posiciones pueden ser largas (comprador) o cortas (vendedor).
\end{itemize}

\section*{Pago y Margen}

No hay costo inicial al celebrar un contrato forward o de futuros, excepto por las comisiones. Sin embargo, en los contratos de futuros, el margen es una garantía que se segrega, pero no se transfiere entre las partes hasta la liquidación.

\section*{Gráficos de Pago}

Los gráficos de pago en forwards y futuros son lineales, ya que no hay costo inicial. En contratos de opciones, este no es el caso. Aquí un ejemplo de cómo funcionaría un pago en un contrato forward:

\begin{itemize}
    \item Si el precio al contado al vencimiento (T) es mayor que el precio acordado, el comprador gana y el vendedor pierde.
    \item Si el precio al contado es menor, el comprador pierde y el vendedor gana.
\end{itemize}

\textbf{Ejemplo}: Supongamos un forward para comprar un activo a \$100 en tres meses:

\begin{itemize}
    \item Si el precio al contado en tres meses es de \$150, el comprador ganará \$50 (ya que puede vender el activo inmediatamente a \$150).
    \item Si el precio es de \$75, el comprador pierde \$25.
\end{itemize}

\section*{Riesgos y Oportunidades}

Los derivados, al ser instrumentos financieros de suma cero, implican que la ganancia de una parte es la pérdida de la otra. No se crea riqueza, solo se transfiere. Por esto, los inversores deben estar preparados para asumir las posibles pérdidas al entrar en posiciones derivadas.

\section*{Conclusión}

Los derivados ofrecen mecanismos eficientes para la gestión de riesgos, pero requieren un entendimiento profundo tanto de los instrumentos como de los mercados en los que se negocian. Para inversores institucionales, los mercados OTC proporcionan mayor flexibilidad, mientras que los inversores más conservadores pueden preferir los mercados organizados debido a su regulación y seguridad.

\end{document}
