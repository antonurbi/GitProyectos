\documentclass[a4paper,12pt]{article}
\usepackage{graphicx}
\usepackage{amsmath}
\usepackage{hyperref}
\usepackage[utf8]{inputenc}

\title{Análisis de la Situación Económica Global y Estrategia de Inversión}
\author{Nombre del Autor}
\date{\today}

\begin{document}

\maketitle

\section{Introducción}
En este informe analizamos la situación económica global actual, tomando como referencia un análisis reciente sobre la globalización y las tensiones comerciales entre Estados Unidos y China. Además, se presenta una estrategia de inversión poco convencional pero potencialmente rentable, basada en la dinámica de los aranceles y el aumento de precios de productos de consumo diario.

\section{Contexto Económico Global}
A pesar de las tensiones geopolíticas y las percepciones de desglobalización, los datos muestran que la globalización sigue siendo un motor clave en la reducción de la desigualdad global. Países emergentes, como China, han experimentado un crecimiento económico significativo, sacando a millones de personas de la pobreza extrema.

Además, la posible imposición de aranceles elevados por parte de EE.UU. a productos importados de China podría desencadenar un aumento en los precios de bienes de consumo, afectando especialmente a los sectores de productos esenciales.

\section{Estrategia de Inversión}
Dada esta situación, proponemos la siguiente estrategia de inversión:

\subsection{Inversión en Empresas Productoras de Bienes Sustitutos Locales}
Ante la posibilidad de un aumento en los aranceles a productos chinos, recomendamos invertir en empresas estadounidenses que producen bienes sustitutos. Los sectores de interés incluyen:
\begin{itemize}
    \item Material escolar
    \item Electrónica de consumo asequible
    \item Ropa de gama media
\end{itemize}

\subsection{Visión de Mediano Plazo}
Esta estrategia debe ejecutarse con una visión de mediano plazo (6-18 meses), esperando que el aumento de la demanda de productos locales impulse las ventas y valoraciones de estas empresas.

\section{Conclusión}
La globalización sigue siendo un fenómeno clave en la economía global, y aunque existan tensiones comerciales, estas también abren oportunidades para inversionistas atentos a las dinámicas del mercado local. La inversión en empresas que puedan beneficiarse de los cambios en la política arancelaria podría ser altamente rentable en el contexto actual.

\end{document}
